\documentclass[aps,twocolumn,groupedaddress,superscriptaddress,floatfix]{revtex4}
\usepackage{amssymb}
\usepackage{graphicx}
\begin{document}
\title{Systematic study of direct photon production in Cu+Cu and Au+Au
collisions at RHIC energies}

\author{D. d'Enterria}
\author{D. Peressounko}
\begin{abstract}
We analyse direct photon production in Cu+Cu and Au+Au collisions
at $\sqrt{s_{NN}}$=62 and 200 GeV within hydrodynamic model. We
consider three different equations of state: EOS of pure hadron
gas, bag motivated EOS with first order phase transition and
lattice-like EOS.  In all cases confronting effective slopes of
thermal photons and final particle multiplicity we extract number
of effective degrees of freedom and compare it with input one.
\end{abstract}
\maketitle

\section{Introduction}
Direct photons provide reasonable estimate of the initial
temperature reached in the collision. Confronting them with other
thermodynamic quantities such as entropy or energy density gives
the possibility to extract number of effective degrees of freedom.

\section{Hydrodynamic model}
To describe evolution of the nucleus-nucleus collision we use
hydrodynamic model plus pQCD contribution scaled with number of
binary collisions. Detailed description of this model one can find
in the \cite{DDE-DP-1}. Hydrodynamic equations being just
equations of local conservation of energy-momentum should be
accomplished by equation of state of hot matter and initial
conditions. We consider these ingredients in details in the next
sections.


\subsection{Equation of state}

Hydrodynamic equations should be accomplished with an equation of
state i.e. dependence
\begin{equation}\label{vs-def}
v_s^2(e)= \left ( \frac{\partial p}{\partial e}\right )_{S},
\end{equation}
where subscript $S$ means that the derivative should be taken for
an adiabatic process:
$$
\frac{\sigma}{n}=const.
$$
Here $\sigma$ is the entropy density and $n$ is the particle
density for the case of gas with conserved number of particles or
some charge (baryon, strange, electric charge etc.) density in the
case of a gas in chemical equilibrium.

In the case of chemically equilibrated gas both energy density and
pressure depends only on temperature and thus
\begin{equation}\label{vs-eqil}
v_s^2=  \frac{\left (\frac{\partial p}{\partial T}\right )}{\left
(\frac{\partial e}{\partial T}\right )}.
\end{equation}

Let us first consider several simple model equations of state and
then construct our more or less realistic one. For massless
particles $p=1/3\varepsilon$ and thus such equation of state
results in constant {\it maximal possible} speed of sound
$v_s^2=1/3$ -- see thin black solid line in the
fig.~\ref{fig:EOS}. For a pure pion gas in chemical equilibrium we
find that the speed of sound monotonically increases with the
temperature and at $T\sim m_\pi$ goes already close to the speed
of sound of massless gas. Similarly, speed of sound in the gas,
consisting of baryons, monotonically increases with temperature
(dashed line in the fig.~\ref{fig:EOS}), but in the case of
mixture of pions and baryons (dotted line in fig.~\ref{fig:EOS})
speed of sound is no more monotonic function: it increases at
$T\ll m_\pi$, decreases at $m_\pi<T<m_p$ and then further
increases, tending to the speed of sound of massless gas.
Increasing of the number of heavier resonances in the gas results
in further decrease of speed of sound -- see dash-ditted line in
fig.~\ref{fig:EOS}. For comparison we present as well results of
the lattice calculations of the speed of sound \cite{latt} in hot
matter. In contrast to the case of hadron resonance gas lattice
$v_s$ has monotonic dependence on the temperature (we fixed in
this plot $T_c=160$~MeV). In addition, this increase is much
steeper than those in the case of pure pion or baryon gas, what
means rapid creation of considerable number of relatively light
degrees of freedom after phase transition. We found, that lattice
predictions of $v_s$ are qualitatively differ from HRG model at
$T<T_c$. This discrepancy probably can be related  to the
unphysically large quark masses used in the lattice calculations
and thus to the absence of light pion states. So, in our analysis
we will not use lattice results directly, but combine them with
HRG model predictions for $T<T_c$.

\begin{figure}[t!]
\includegraphics[width=\columnwidth]{vs_cheq.eps}
\caption{Speed of sound squared for different gases in chemical
equilibrium.}\label{fig:EOS}
\end{figure}

In this analysis we use three different equations of state: (a)
EOS of resonance hadron gas without phase transition, (b) EOS with
first order phase transition and (c) EOS using lattice
parameterization. The first case (a), EOS of the resonance
hadronic gas is modeled as a gas of non-interacting $\sim$400
known hadrons and hadronic resonances with masses below 2.5
GeV/$c^2$. In the second case (b) EOS of hadron resonance gas EOS
is complimented with first order phase transition to the QGP at
$T_{c}$ = 165 MeV, described within bag model as a gas of massless
quarks and gluons with latent heat $\Delta\varepsilon\approx$ 1.4
GeV/fm$^3$. The QGP is modeled as an ideal gas of massless quarks
($N_f$ = 2.5 flavours) and gluons with total degeneracy
$g_{\mbox{\tiny{\it{QGP}}}} = (g_{\mbox{\tiny{\it{gluons}}}}+7/8\,
g_{\mbox{\tiny{\it{quarks}}}})$ = 42.25. The corresponding EoS,
$p=1/3\varepsilon-4/3B$ ($B$ being the bag constant), has sound
velocity $c_s^2=\partial p/\partial \varepsilon = 1/3$. Both
phases are connected via the standard Gibbs' condition of phase
equilibrium, $p_{\mbox{\tiny{\it{QGP}}}}(T_{c}) =
p_{\mbox{\tiny{\it{HRG}}}}(T_{c})$, during the mixed phase. The
external bag pressure, calculated to fulfill this condition at
$T_c$, is $B\approx$ 0.38 GeV/fm$^3$. In the case (c) we construct
our equation of state as following. First we compare dependence
$e(T)$ for HRG and lattice and find some point, where
\begin{eqnarray}\label{Conn}
 e_{HRG}(T_{conn})&=&e_{latt}(T_{conn},T_c)  \\
 \frac{
 de_{HRG}}{dT}(T_{conn})&=&\frac{de_{latt}}{dT}(T_{conn},T_c).
\end{eqnarray}
Here we consider transition temperature $T_c$ and connection
temperature $T_{conn}$ as free parameters. We find $T_c=168$ MeV
and $T_{conn}=182$~MeV. Having $T_c$ fixed we then find point,
where \begin{equation}\label{vs-conn}
 v_s^{HRG}(T_{vs})=v_s^{Latt}(T_{vs}). \nonumber
\end{equation}
We find $T_{vs}=172$~MeV. Resulting dependencies of $e(T)$ and
$v_s(T)$ are presented in the Fig.~\ref{fig:E-vs}.


\begin{figure*}[tbh]
\includegraphics[width=\columnwidth]{E.eps}
\includegraphics[width=\columnwidth]{vs.eps}
\caption{Normalized energy density (left) and speed of sound
(right) for equations of states. Black line represents equation of
state of chemically equilibrated resonance hadron
gas.}\label{fig:E-vs}
\end{figure*}


Energy density is less sensitive to the presence of the light
states, while speed of sound is much more sensitive to the mass
state distribution.

\subsection{Initial conditions}
Second important ingredient of the hydrodynamic modes is initial
conditions. We use the same procedure as in \cite{DDE-DP-1}. For
different colliding systems we use different initial conditions,
summarized in Table~\ref{tab:IC}

\begin{table*}
\caption{Initial conditions for different EOSes and colliding
systems}\label{tab:IC}
\begin{tabular}{|c|c|c|c|c|}
  \hline
  EOS  & Au+Au, 200                     & Au+Au, 62   & Cu+Cu, 200 &  Cu+Cu, 62 \\
  \hline
  Bag  & $\tau_{in}=0.15$, $E_{in}=220$ & $\tau_{in}=0.45$, $E_{in}=33$ & $\tau_{in}=0.1$, $E_{in}=160$ & $\tau_{in}=0.30$, $E_{in}=24$ \\
  Latt & $\tau_{in}=0.15$, $E_{in}=220$ & $\tau_{in}=0.45$, $E_{in}=36.5$ &  $\tau_{in}=0.10$, $E_{in}=170$  & $\tau_{in}=0.30$, $E_{in}=30.5$ \\
  HRG  & $\tau_{in}=0.60$, $E_{in}=37$  & $\tau_{in}=0.60$, $E_{in}=22.5$ & $\tau_{in}=0.60$, $E_{in}=17$ & $\tau_{in}=0.60$, $E_{in}=12$ \\
  \hline
\end{tabular}
\end{table*}

We check that we properly reproduce charged particles yield in
midrapidity comparing our yield with data measured with PHOBOS
RHIC experiment and other RHIC experiments where available, see
fig.~\ref{fig:Nch}. One can see that centrality dependence both of
the total charged particles yield and of the transverse energy are
reproduced well for all considered EOSes and curves go close to
each other.

\begin{figure*}[t]
\includegraphics[width=\columnwidth]{Nch_Au200.eps}
\includegraphics[width=\columnwidth]{Et_Au200.eps}
\includegraphics[width=\columnwidth]{Nch_Au62.eps}
\includegraphics[width=\columnwidth]{Et_Au62.eps}
\includegraphics[width=\columnwidth]{Nch_Cu200.eps}
\includegraphics[width=\columnwidth]{Et_Cu200.eps}
\includegraphics[width=\columnwidth]{Nch_Cu62.eps}
\includegraphics[width=\columnwidth]{Et_Cu62.eps}
\caption{Normalized charged particles density and transverse
energy density calculated with different EOSes.}\label{fig:Nch}
\end{figure*}


\section{Number of effective degrees of freedom}
Correlating direct photon effective temperatures and total charged
particles yields we constructed numbers of effective degrees of
freedom and compared them to ideal ones implemented into
hydrodynamics, see fig.~\ref{fig:Hdf}

\begin{figure*}[t]
\includegraphics[width=0.66\columnwidth]{Ndf_Au200_bag.eps}
\includegraphics[width=0.66\columnwidth]{Ndf_Au200_Latt.eps}
\includegraphics[width=0.66\columnwidth]{Ndf_Au200_HRG.eps} \\
\includegraphics[width=0.66\columnwidth]{Ndf_Au62_bag.eps}
\includegraphics[width=0.66\columnwidth]{Ndf_Au62_Latt.eps}
\includegraphics[width=0.66\columnwidth]{Ndf_Au62_HRG.eps} \\
\includegraphics[width=0.66\columnwidth]{Ndf_Cu200_bag.eps}
\includegraphics[width=0.66\columnwidth]{Ndf_Cu200_Latt.eps}
\includegraphics[width=0.66\columnwidth]{Ndf_Cu200_HRG.eps} \\
\includegraphics[width=0.66\columnwidth]{Ndf_Cu62_bag.eps}
\includegraphics[width=0.66\columnwidth]{Ndf_Cu62_Latt.eps}
\includegraphics[width=0.66\columnwidth]{Ndf_Cu62_HRG.eps} \\
\caption{Effective number of degrees of freedom}\label{fig:Ndf}
\end{figure*}




\begin{thebibliography}{99}
\bibitem{DDE-DP-1}D.d'Enteriia and D.Peressounko,
\bibitem{latt}S.D.~Katz, Plenary talk at 18th International Conference on Ultrarelativistic Nucleus-Nucleus
Collisions: Quark Matter 2005 (QM 2005), Budapest, Hungary, 4-9
Aug 2005; e-Print Archive: hep-ph/0511166.
\bibitem{Nch-PHOBOS}G.Roland et al.,(PHOBOS collaboration),
nucl-ex /0510042.
\end{thebibliography}
\end{document}
% ------------------------------------------------------------------------
